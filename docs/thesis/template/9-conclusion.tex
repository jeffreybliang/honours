\chapter{Concluding Remarks}\label{chap:conclusion}

If you wish, you may also name that section \emph{``Conclusion and Future Work''}, though it might not be a perfect choice to have a section named ``A \& B'' if it has subsections ``A'' and ``B''. Also note that you don't necessarily have to use these subsections; that also depends on how much content you have in each. (E.g., having a section header might be odd if it contains just three lines.)


\section{Conclusion}

This section usually summarizes the entire paper including the conclusions drawn, e.g., did the developed techniques work? Maybe add why or why not. Also don't hold back on limitations of your work; it shows that you understood what you have done. And science isn't about claiming how great something is, but about objectively testing hypotheses. Also note that every single scientific paper has such a section, so you can check out many examples, preferably at top-tier venues, e.g., by your supervisor(s).


\section{Future Work}

On top of that, you could discuss future work (and make clear why that is future work, i.e., by which observations did they get justified?).

Note that future work in scientific papers is often not mentioned at all or just in a very few sentences within the conclusion. That should not stop you from putting some effort in. This will (also) show the examiner(s)/supervisor(s) how well you understood the topic or how engaged you are.
